\documentclass{scrreprt}		% Keine doppelte Überschrift bei chapters

\usepackage{geometry}			% Formatierung der Seitenränder
\geometry{
	a4paper, 
	top=20mm, 
	left=20mm, 
	right=20mm, 
	bottom=20mm, 
	headsep=10mm, 
	footskip=12mm
}

\usepackage[utf8]{inputenc}		% Ermöglicht nahezu alle Zeichencharakter ohne sonderzeichen Kennzeichnung
\usepackage[ngerman]{babel}
\usepackage{textcomp}
\usepackage{xcolor} 			% Ermöglichen von verschiedenen Farben z. B. Hyperlinks
\usepackage{color} 
\usepackage{colortbl}
\usepackage{makecell}

% Einrichten von Hyperlins innerhalb des PDF Dokuments
\usepackage[
	pdfpagelabels, 
	bookmarksopen=true, 
	bookmarksnumbered=true, 
	linkcolor=black, 
	hypertexnames=false
] {hyperref} 

\usepackage{graphicx}			% Einbinden von Grafiken

\usepackage[final]{pdfpages}
\usepackage{array}
\usepackage{upgreek}
\usepackage{amsmath}			% Mathematische Funktionalitäten
\usepackage{amssymb}			% Mathematische Symbole
\usepackage{amsfonts}
\usepackage{leftidx}
\usepackage{caption,pdfpages}

\usepackage{capt-of}			% Überschriften bei minipages

\usepackage{wrapfig}			% Text um Grafik wrappen
\usepackage{subcaption}

\usepackage{booktabs}
\usepackage{tabularx}

\usepackage[footnote,nohyperlinks]{acronym}	% Ermöglicht das Abkürzungsverzeichnis, darstellung der langform als footnote
% --- Weitere Optionen --- %
% footnote 		-- 	die Langform als Fußnote ausgeben
% nohyperlinks 	-- 	wenn hyperref geladen ist, wird die Verlinkung unterbunden
% printonlyused -- 	nur Abkürzungen auflisten, die tatsächlich verwendet werden.
						% Im printonlyused-Modus kann zusätzlich noch die Option withpage verwendet werden. Hierdurch wird im Abkürzungsverzeichnis zusätzlich die Seitenzahl, auf welcher die Abkürzung als erstes verwendet wurde, ausgegeben.
% smaller 		-- 	Text soll kleiner erscheinen, das Paket relsize wird vorausgesetzt
% dua 			-- 	es wird immer die Langform ausgegeben
% nolist 		-- 	es wird keine Liste mit allen Abkürzungen ausgegeben
% --- % Referenz: https://de.wikibooks.org/wiki/LaTeX-Wörterbuch:_Abkürzungsverzeichnis %% Stand 12.08.15

\setlength{\parindent}{0pt}	% Entfernen von automatischen Texteinrückungen nach einem Absatz
