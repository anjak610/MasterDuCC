\addchap{Zusammenfassung}


In dieser Arbeit ist ein Szenenrekonstruktionsalgortithmus für stereoskopische Bildaufnahmen entstanden. Dieser ist mit zuvor bekannten intrinsischen Kameraparametern im Stande die extrinsischen Kameraparameter zu bestimmen und daraufhin die 3D-Szene zu rekonstruieren. Des Weiteren unterstützt der Algorithmus die Verwendung von unterschiedlich aufgelösten Bildern. \\

Für die Entwicklung des Algorithmus wurde ein synthetischer Szenenaufbau implementiert. Die grundlegenden Funktionen für die Bestimmung der extrinsischen Kameraparameter und der darauf folgenden Rekonstruktion der Szene, wurden anhand eines synthetisch generierten Szenenaufbaus entwickelt und validiert. der so entstandene Algorithmus konnte dann an realen Stereoaufnahmen angewandt werden. Die auf Grund ungenauer Punktekorrespondenzen entstandenen Fehler, konnten mit Hife des synthetischen Beispiel simuliert und somit lokalisiert werden. Mögliche Lösungsansätze wurden dann mit Zuhilfenahme von Literaturquellen entwickelt und der Algorithmus entsprechend modifiziert. \\

Im nächsten Abschnitt wurde auf den in vielen Computer-Vision-Applikationen genutzten Ansatz für eine effiziente Szenenrekonstruktion, welcher keine zusätzliche Kamerakalibrierung beinhaltet, eingegangen. Der Ansatz basiert auf der vorherigen Rektifizierung von Stereoaufnahmen, was eine Vereinfachung der Korrespondenzanalyse mit sich bringt. Anhand dieser Korrespondenzen können Tiefenkarten erstellt werden, die eine direkte Abschätzung der Szenentiefe darstellt. Für diese Applikationen werden meist gleiche Kameraauflösungen vorausgesetzt. In dieser Arbeit wurde ein Rektifizierungsalgoritmus implementiert und auf Bildquellen verschiedener Auflösungen angewandt. Somit konnte die Funktionsweise des Algorithmus analysiert und Mögliche Erweiterungen dessen für solche Anwendungen entwickelt werden. 

\addchap{Ausblick}

Nachdem die ersten funktionsfähigen prototypischen Algorithmen für eine Szenenrekonstruktion implementiert sind, sollen nun zwei Möglichkeiten für dessen Weiterentwicklung in zukünftigen Projekten aufgezeigt werden.\\

Zum einen  



hauptskript ist soweit funktionsfähig 

noch das ganze biild durchgehen rekonstruieren. 
komplette punktekoorepondenzen

wegkommen von den inputparametern kann aber bei unterschiedliche pixelvergrößerungen bei der rekonsturktion probleme macht....maß für pxellänge durch vllt schachbrett weil da distanz beaknnt.

rechenleitung sparen 



Sowohl den ersten als auch den zweiten Ansatz dahingehend weiter verfolgen ihn an einem Beispiel einer hochauflösenden rgb kamera und einer standard infrarot kamera auszuprobieren und die Ergebnisse beider zu vergleichen. 

Den ersten Ansatz kann noch dahingehend alternativ aufgebaut werden, dass die intrinsischen Kameraparameter nicht mehr benötigt werden, dazu gäbe es einen sogenannten startified approch, welcher die Rekonstruktion der anahnd der Fundamentalmatrix erlaubg, jedoch nur bis auf eine rojektive komponente...

