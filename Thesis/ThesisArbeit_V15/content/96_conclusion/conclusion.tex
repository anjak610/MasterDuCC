\chapter{Fazit}



%Outlook ist dann: Zusammenfassung der Arbeit, sprich was wurde gemacht, welche ergebnisse wurden erziehlt. Und  ganz zum schluss besprechung der offenen Stellen. Was macht Probleme, wie kann man es evtl beheben.
%
%
%
%Das im Vorfeld angegebene Ziel, einen Szenenrekonstruktionsalgorithmus zu entwickeln , der...... wurde erreicht.\\
%
%Der aufbau des Algorithmus wurde ahnahnd eines virtuellen und eines reellen Beispiels beschrieben und die Ergebnisse der Test aufgezeigt.\\
%
%
%Besprechung offener Stellen:
%
%Vermeidung von vorwissen der intrinsischen Kameraparameter durch stratified approach


In dieser Arbeit ist ein Szenenrekonstruktionsalgortithmus für stereoskopische Bildaufnahmen entstanden. Dieser ist mit zuvor bekannten intrinsischen Kameraparametern im Stande die extrinsischen Kameraparameter zu bestimmen und daraufhin die 3D-Szene zu rekonstruieren. 

Dabei ist es möglich bei den Stereoaufnahmen mit unterschiedlichen Auflösungen 

Hierfür wurde zunächst in einem synthetischen Beispiel die Grundlegende




Des weiteren wurde ein Ansatz für eine schnelle Rekosntruktion implementiert