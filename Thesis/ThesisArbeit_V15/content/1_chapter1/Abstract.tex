\addchap{Abstract}

Das Erkennen und rekonstruieren von dreidimensionalen Objekten aus zweidimensionalen Bildquellen ist Kern aktueller Forschungsbereiche wie dem autonomen Fahren oder der Entwicklung von hochpräzisen medizinische Navigationsystemen. Das Ziel ist es mit Hilfe weniger Kamerabilder einer Szene eine möglichst präzise, dreidimensionale Rekonstruktion dieser Szene zu erhalten.\\

In dieser Arbeit wurde ein Szenenrekonstruktionsalgorithmen für stereoskopische Bildaufnahmen implementiert. Dabei sollte die Frage geklärt werden, ob eine Szenenrekonstruktion mit Kameras unterschiedlicher Auflösungen möglich ist. Somit könnten beispielsweise Applikationen zur Szenenrekonstruktion für Drohnen geschrieben werden, welche mit einer hochauflösenden rgb Kamera und einer Standard Infrarotkamera ausgestattet sind. Des Weiteren soll der Algorithmus die extrinsischen Kameraparameter bestimmen können. Der unter diesen Bedingungen entstandene Algorithmus ist in der Lage bei bekannten intrinsischen Kameraparametern aus Punktekorrespondenzen zweier Bildern unterschiedlicher Auflösung extrinsische Kameraparameter zu bestimmen und die 3D-Szene über ein Triangulationsverfahren zu rekonstruieren. Der Schwerpunkt dieser Arbeit liegt auf der Entwicklung und Umsetzung des prototypischen Programms. Die Grundlegenden Funktionen des Algorithmus wurden in einem synthetischen Beispiel implementiert und später für die Verwendung von reellen Bilddaten entsprechend erweitert. \\

Des Weiteren ist ein Ansatz für einen zweiten Algorithmus entstanden, welcher das Ziel verfolgt mit wenig vorab Informationen eine Szene Rekonstruieren zu können. Der Ansatz basiert auf einer zuvorigen Rektifizierung des Stereobildpaares, welche die Suche nach korrespondierenden Punkten erleichtert. Anhand dieser Punkte kann eine Tiefenkarte erstellt werden, welche relaitv schnell und ohne großen rechenaufwand ein Maß für die Tiefe der 3D-Szene gibt.\\

Als Zusatz ist in dieser Arbeit noch ein Algorithmus für die Ermittlung der Punktekorrespondenzen zwischen den Stereoaufnahmen zweidimensionaler Schachbretter entwickelt worden. Dieser ist im Stande aus einer Liste aus Schachbretteckpunkten die Reihen und Spalten des Schachbretts zu rekonstruieren und die Eckpunkte eindeutig zu identifizieren.




%die 3D-Szene durch Erstellen einer Tiefenkarte 
%
%Der Ansatz basiert auf der vorherigen Rektifizierung des Stereobildpaares für eine korresponde
%
% Der zweite Algorithmus beinhaltet den Ansatz eine Szene basierend auf einer zuvorigen Rektifizierung der Stereoaufnahmen zu rekonstruieren. Vorteil dieser Methode sind die wenigen benötigten Informationen. Lediglich die Punktekorrespondenzen und die Fundamentalmatrix müssen bekannt sein. Des Weiteren bietet er die Möglichkeit über eine Tiefenkarte die 3D Szene zu rekonstruieren. Jedoch müssen Bilder unterschiedlicher Auflösungen zuvor algorithmisch angepasst werden.\\
% 



%
%
%Rekonstruktion durch Rektifizierung eines Stereobildpaares, wie es in vielen Computer-Vision-Applikationen angewandt wird, analysiert und getestet. Dabei wurde ein Rektifizierungsansatz nach \textit{Zhengyou Zhang} implementiert und die Auswirkung von unterschiedlichen Kameraauflösungen auf die Ergebnisse der Rektifizierung beobachtet.  
%
%
%
%Die Szene wird in diesem Ansatz über ein Triangulationsverfahren 





%
%Der Algorithmus wurde anhand eines synthetisch erstellten Stereoaufbaus entwickelt und validiert. 
%
%
% Das Ergebnis wurde dann an reellen Stereoaufnahmen zweier Kameras getestet und entsprechend erweitert. Das Resultat ist ein funktionsfähiger Algorithmus, welcher aus Punktekorrespondenzen in einem Stereobildpaar mit unterschiedlichen Auflösungen extrinsische Kameraparameter bestimmen und eine 3D-Szenen rekonstruieren kann. 
%
%
%
%Bei den Ergebnissen der Rekonstrukion ist mit Ungenauigkeiten zu rechnen, da Auswirkungen von Bildfehlern, wie Rauschen, durch Näherungsverfahren minimiert wurden.\\
%
%%In dieser Arbeit wurde ein Szenenrekonstruktionsalgorithmus entwickelt, welcher sowohl in der Lage sein soll die extrinsischen Kameraparameter zu bestimmen als auch 3D-Szenen aus Bildquellen unterschiedlicher Auflösung zu rekonstruieren. 
%
%
%Abschließend wurde ein Verfahren der schnellen Rekonstruktion durch Rektifizierung eines Stereobildpaares, wie es in vielen Computer-Vision-Applikationen angewandt wird, analysiert und getestet. Dabei wurde ein Rektifizierungsansatz nach \textit{Zhengyou Zhang} implementiert und die Auswirkung von unterschiedlichen Kameraauflösungen auf die Ergebnisse der Rektifizierung beobachtet.  
%


