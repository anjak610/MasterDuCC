\chapter{Einleitung}
\label{sec:einleitung} 




Die Computer Vision ist ein Fachbereich der Computer Science mit dem Fokus auf die Entwicklung von künstlicher Intelligenz, die ein visuelles Verständnis ihrer Umgebung besitzen. Folglich wird in der Computer Vision der Weg von visuellen Eindrücken oder Bildern aus der Realität in den Rechner beschrieben \cite{ComputerVision}. Der Mensch ist mit der Fähigkeit ausgestattet, gesehene Bilder zu verarbeiten und  kann die ihn umgebene Welt verstehen. Maschinen, die eine ähnliche Fähigkeit besitzen, wären somit ebenfalls in der Lage Entscheidungen auf Grund von visuellen Eindrücken zu fällen. Das entwickeln solcher Maschinen und den damit verbundenen Grundprinzipien und Programmen sind die Forschungsmittelpunkte von aktuellen Anwendungsbereichen wie dem Autonomen Fahren, Motion-Capturing, Bewegungserkennungen oder Service Robotern.\\

In dieser Masterarbeit wurde ein Algorithmus zur Rekonstruktion einer Szene aus stereoskopischen Bildquellen entwickelt. Das typische Verfahren einer Stereorekonstruktion basiert auf den Grundbausteinen Bildaufnahme und Bildanalyse\cite{ComputerVision}. In der Bildaufnahme wird eine Szene oder ein Objekt mit Hilfe von Kameras, Sensoren oder Lasern aufgenommen und als digitale zweidimensionale Bilder an den Computer weitergegeben. In der Bildanalyse werden die aufgenommenen Bilder ausgewertet um so die dreidimensionale Szene rekonstruieren zu können. Für die Analyse ist es von Vorteil die Kameraparameter, wie Position und Auflösung, zu kennen. Sind diese jedoch nicht bekannt, können die Bildquellen genutzt werden um die Kameraparameter abzuschätzen. Eine solche Abschätzung wird als Kamerakalibrierung\cite{HZ,Ferid,Elements,ZZGXr} bezeichnet. Die Position und Rotation einer Kamera im Raum werden als die extrinsischen Kameraparameter bezeichnet. Parameter wie die Auflösung oder Brennweiten, werden als die intrinsischen Kameraparameter bezeichnet\cite{HZ,Ferid}. Im Zuge dieser Arbeit ist ein Algorithmus entstanden, welcher im Stande ist die extrinsischen Kameraparameter bei Kameras gleicher und unterschiedlicher Auflösung zu bestimmen und eine 3D-Szenenrekonstruktion durchzuführen. Der in dieser Arbeit entwickelte Algorithmus wurde mithilfe eines synthetisch erstellten Beispiels verifiziert und auf eine reelle Szenenaufnahme angewandt. Beim Entwickeln von Algorithmen für Computer Vision Applikationen sieht man sich mit komplizierten Aufgaben und Herausforderungen konfrontiert. Bei der Aufnahme von Bildern, kann es  zu unvorhersehbaren Bildfehlern wie beispielsweise Rauschen oder Verzerrungen durch die Kameralinse kommen, was nicht oft zum Verlust von Referenzdaten führt.\\

In der synthetischen Rekonstruktion wird zuerst eine 3D-Szene in zwei voneinander unterschiedlich positionierten, simulierten Kameras projiziert um virtuelle Bilddaten zu generieren. Anhand dieser 2D-Bilddaten wird die Kamerakalibrierung getestet. In der synthetischen Rekonstruktion werden die Werte für Auflösung und Brennweite selbst gesetzt. Im Test des Algorithmus mit reellen Bilddaten, wird für dessen Schätzung auf ein bereits existierendes Programm zurückgegriffen. Die intrinsischen Parameter werden mit dem in dieser Thesis entwickelten Algorithmus für die Schätzung der Positionen und Orientierungen der Kameras kombiniert, um die extrinsischen Kameraparameter anhand der systhetischen Daten zu schätzen.  Die durch die Schätzung erhaltenen Kameraparameter können im synthetischen Beispiel mit den zuvor definierten Parametern verglichen werden, um den Algorithmus zu verifizieren. Diese Kameraparameter werden im entwickelten Rekonstruktionsalgorithmus dazu verwendet, die ursprüngliche 3D-Szene wieder herzustellen und die Funktionsweise der Rekonstruktion zu analysieren. Anschließend wird der Algorithmus an einem realen Stereobildpaar getestet und die entsprechenden Modifizierungen für die Arbeit mit realen fehleranfälligen Bilddaten genau aufgezeigt. \\

Ein zweiter Ansatz für einen Szenenrekonstruktionsalgorithmus aufgezeigt. Dieser ist auf eine schnelle und effiziente Rekonstruktion einer Szene ausgelegt und weniger auf die Bestimmung von Kameraparametern. Die Methode basiert auf einer zuvorigen Rektifizierung der Bilder, um Punktekorrespondenzen effizient zu ermitteln. Anhand dieser Punktekorrespondenzen kann eine Tiefenkarte der 3D-Szene erstellt werden. Auf diese Weise ist es möglich eine effektive Abschätzung für die Tiefe einer Szene zu bekommen. Die meisten kommerziellen Rekonsturkions-Applikationen setzten gleiche intrinsische Kameraparameter voraus. Um zu testen welche Auswirkungen eine Rektifizierung von Bildern unterschiedlicher Auflösungen hat, wird ein Rektifizierungsalgorithmus nach \textit{Charles Loop \& Zhengyou Zhang} implementiert. In Kapitel \ref{sec:rectification} wird der implementierte Algorithmus auf Abbildungen mit unterschiedlichen Auflösungen angewandt und die Auswirkungen unterschiedlicher Kameraauflösungen analysiert. \\


In Kapitel \ref{sec:schachbrettAlg} wird ein weiterer Algorithmus vorgestellt. Dieser wurde implementiert um die detektierten Eckpunkte eines Schachbretts zu sortieren und eindeutig zu identifizieren und kann für die Korrespondenzanalyse bei Stereoaufnahmen von Schachbrettern mit starker optischen Verzerrung verwendet werden. 



%\textcolor{red}{(aktualisieren schreiben!!!)}
%Die Kapitel 1 bis 5 umfassen die theoretischen mathematischen Hintergründe, welche für das Verständnis der implementierten Kalibrierungs- und Rekonstruktionsalgorithmen vorhanden sein müssen. Kapitel 6 befasst sich dann umfassend mit dem für die virtuelle Rekonstruktion entwickelten Algorithmus für die Kamerakalibrierung und Szenenrekonstruktion. Kapitel 7 befasst sich dann mit der Kalibrierung und Rekonstruktion, wenn die virtuellen Kameras unterschiedliche Auflösungen aufweisen. In Kapitel 8 wird der entwickelte Algorithmus auf reale Bilddaten angewandt. In Kapitel 9 werden die Auflösungen der realen stereoskopischen Bilder künstlich verändert und die Ergebnisse verifiziert. Kapitel 10 beschreibt einen Algorithmus zur Sortierung von zuvor detektierten Eckpunkten eines Schachbretts, welcher im Zuge der Korrespondenzanalyse von Bildpunkten in Stereoskopischen Aufnahmen entstanden ist.
%%Kapitel 6 beinhaltet die einzelnen Schritte des in Abbildung \ref{fig:ArbeitsProzessMinimal} aufgezeigten Kalibrierungs- und Rekonstruktionsalgorithmus des Minimalbeispiels. Mit diesem Kapitel werden die zuvor beschriebenen mathematischen Werkzeuge zusammengefasst und in Bezug der Stereoskopischen Szenenrekonstruktion angewandt. Der Grund der implentierung dieses Minimalbeispiels war unter anderem der, dass somit Fehler, welche im Realbeispiel aufgetreten sind rekonstruiert und eine mögliche Lösung dieser ermittelt werden konnte. 


