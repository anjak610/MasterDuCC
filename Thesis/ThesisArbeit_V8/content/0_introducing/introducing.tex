\chapter{Einleitung}
\label{sec:einleitung} 



Die Computer Vision ist ein Fachbereich der Computer Science mit dem Fokus auf der Enwicklung von künstlicher Intelligenz, die ein visuelles Verständnis ihrer Umgebung besitzen. Folglich wird in der Computer Vision der Weg von visuellen Eindrücken oder Bildern aus der Realität in den Rechner beschrieben \cite{ComputerVision}. Der Mensch ist mit der Fähigkeit ausgestattet, gesehene Bilder zu verarbeiten und  kann die ihn umgebene Welt verstehen. Maschinen, die eine ähnliche Fähigkeit besitzen, wären somit ebenfalls in der Lage Entscheidungen auf Grund von visuellen Eindrücken zu fällen. Das entwickeln solcher Maschinen und den damit verbundenen Grundprinzipien und Programme sind die Forschungsmittelpunkte von aktuellen Anwendungsbereichen wie dem Autonomen Fahren, Motion-Caturing, Bewegungserkennungen oder Service Robotern.\\

In dieser Masterarbeit wurde ein Algorithmus zur Rekonstruktion einer Szene aus stereoskopischen Bildquellen entwickelt, welcher auch die Möglichkeit von unterschiedlichen Bildauflösungen zwischen den Bildquellen in Betracht zieht. Das typische Verfahren einer Stereorekonstruktion basiert auf den Grundbausteinen, Bildaufnahme und Bildanalyse\cite{ComputerVision}. In der Bildaufnahme wird eine Szene oder ein Objekt mit Hilfe von Kameras, Sensoren oder Lasern aufgenommen und als digitale zweidimensionale Bilder an den Computer weitergegeben. In der Bildanalyse, werden die aufgenommenen Bilder ausgewertet um so die dreidimensionale Szene rekonstruieren zu können. Für die Analyse ist es essentiell die Kameraparameter, wie Position und Auflösung, zu kennen. Sind diese jedoch nicht bekannt, können die Bildquellen genutzt werden um die Kameraparameter abzuschätzen. Eine solche Abschätzung wird als wird als Kamerakalibrierung\cite{HZ,Ferid,Elements,ZZGXr}. Die Position und Rotation einer Kamera im Raum werden als die extrinsischen Kameraparameter bezeichnet, Parameter wie die Auflösung oder Brennweiten, werden als die intrinsischen Kameraparameter bezeichnet\cite{HZ,Ferid}. Im Zuge dieser Arbeit ist ein Algorithmus entstanden, welcher unter anderem im Stande ist die Kameras gleicher und unterschiedlicher Auflösung zu kalibrieren eine 3D-Szenenrekonstruktion durchzuführen. Der vollständige Algorithmus wurde mithilfe eines virtuellen Beispiels verifiziert und auf eine reelle Szenenaufnahme angewandt. Mit dem Entwickeln von Algorithmen für Computer Vision Applikationen, sieht man sich mit immer wieder mit komplizierten Aufgaben und Herausforderungen konfrontiert. Bei der Aufnahme von Bildern, kann es immer wieder zu unvorhersehbaren Bildfehlern wie beispielsweise Rauschen oder Verzerrungen durch die Kameralinse kommen, was auch nicht oft zum Verlust von Referenzdaten führt. Im Kapitel \nameref{sec:real} wird aufgeführt, wie mit solchen Fehlern umgegangen werden kann.\\

In der virtuellen Rekonstruktion wird zuerst eine 3D Szene in zwei voneinander unterschiedlich positionierten, simulierten Kameras projiziert um virtuelle Bilddaten zu generieren. Anhand dieser 2D-Bilddaten wird die Kamerakalibrierung getestet. In der virtuellen Rekonstruktion, werden die werte für Auflösung und Brennweite, welche als intrinsische Parameter bezeichnet werden, selbst gesetzt. Im Test des Algorithmus mir reellen Bilddaten, wird für dessen Schätzung auf ein bereits existierendes Prgramm zurückgegriffen. Die intrinsischen Parameter werden mit dem hier entwickelten Algorithmus für die Schätzung der Positionen und Orientierungen der Kameras, die als extrinsische Parameter betzeichnet werden, kombiniert um die Kameras anhand der virtuellen Daten zu kalibrieren. Die durch die Schätzung erhaltenen Kameraparameter können im virtuellen Beispiel so einfach mit den zuvor definierten Parametern verglichen werden, um den Algorithmus zu verifizieren. Diese Kameraparameter werden dann entwickelten Rekonstruktionsalgorithmus dazu verwendet, in die Ursprüngliche 3D-Szene wieder herzustellen und die Funktionsweise der Rekonstruktion zu analysieren. \\

%\textcolor{red}{(aktualisieren schreiben!!!)}
%Die Kapitel 1 bis 5 umfassen die theoretischen mathematischen Hintergründe, welche für das Verständnis der implementierten Kalibrierungs- und Rekonstruktionsalgorithmen vorhanden sein müssen. Kapitel 6 befasst sich dann umfassend mit dem für die virtuelle Rekonstruktion entwickelten Algorithmus für die Kamerakalibrierung und Szenenrekonstruktion. Kapitel 7 befasst sich dann mit der Kalibrierung und Rekonstruktion, wenn die virtuellen Kameras unterschiedliche Auflösungen aufweisen. In Kapitel 8 wird der entwickelte Algorithmus auf reale Bilddaten angewandt. In Kapitel 9 werden die Auflösungen der realen stereoskopischen Bilder künstlich verändert und die Ergebnisse verifiziert. Kapitel 10 beschreibt einen Algorithmus zur Sortierung von zuvor detektierten Eckpunkten eines Schachbretts, welcher im Zuge der Korrespondenzanalyse von Bildpunkten in Stereoskopischen Aufnahmen entstanden ist.
%%Kapitel 6 beinhaltet die einzelnen Schritte des in Abbildung \ref{fig:ArbeitsProzessMinimal} aufgezeigten Kalibrierungs- und Rekonstruktionsalgorithmus des Minimalbeispiels. Mit diesem Kapitel werden die zuvor beschriebenen mathematischen Werkzeuge zusammengefasst und in Bezug der Stereoskopischen Szenenrekonstruktion angewandt. Der Grund der implentierung dieses Minimalbeispiels war unter anderem der, dass somit Fehler, welche im Realbeispiel aufgetreten sind rekonstruiert und eine mögliche Lösung dieser ermittelt werden konnte. 


