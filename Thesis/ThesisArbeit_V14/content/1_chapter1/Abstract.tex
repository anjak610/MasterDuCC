\addchap{Abstract}
 
 Over the last decades computer vision scientist have taken a new approach to vision. They build different computational models of what shoud be computed, what can really be computed, and how these computations can be realized by computer programs, and they use computers to test their models are correct. The result is a better understandung of vision from a different point of view, and at the same time some working artificial vision systems are built that can be used in idustry, medicine, etc. The knowledge obtained on neirophysiology and psychophysics have given hints to and influenced computer vision scientists, helping find solutions to the design of specific algorithms and implementation of vision systems. On the other Hand coputational vision has also given nerophysiologists and psychophysicsist a mathematical framework for modeling vision processes.\cite{ZZGXr}
 
 
 Abstract: wichtigkeit/aktualität von szenenrekonstruktion...  In dieser arbeit wurde ein szenenenrekonstruktionsalgorithnus implementiert und getestet... der implemtierte algorithmus wurde implementiert um bilder verschiedene Kamera-Auflösung zu rekonstruieren...
 Ergebniss: algorithmus funktionier, mit einigen problemen, wobei erste lösungsvorschläge aufgezeigt werden